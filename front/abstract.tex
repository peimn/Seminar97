
% -------------------------------------------------------
%  Abstract
% -------------------------------------------------------


\pagestyle{empty}

\شروع{وسط‌چین}
\مهم{چکیده}
\پایان{وسط‌چین}
\بدون‌تورفتگی
روان نگه داشتن جریان ترافیک فوایدی از جمله جلوگیری از هدر رفت زمان، صرفه‌جویی در سوخت و جلوگیری از آلودگی محیط زیست را شامل می‌شود.
بنابر محدودیت‌های موجود بر توسعه بزرگراه‌ها و غیرمنطقی بودن آن، رویکردهای متفاوتی برای بهبود جریان ترافیک نیاز است.
هوش مصنوعی به رغم پیشرفت‌های اخیر در همه‌ی حوزه‌ها ورود داشته است. الگوریتم‌های متفاوتی به منظور کنترل ترافیک و بهبود جریان آن توسعه یافته‌اند که برپایه الگوریتم‌های شناخته‌شده‌ای مانند یادگیری تقویتی، یادگیری تقویتی عمیق، یادگیری-Q و فرایندهای تصمیم‌گیری مارکوف شکل گرفته‌اند. ماهیت یادگیری تقویتی و یادگیری-Q و عملکرد پاداش و مجازاتی آن‌ها و بهبود عملکرد خود براساس وضعیت‌های قبلی محیط و عامل‌ها، آن را گزینه‌ای مناسب جهت به‌کارگیری در حوزه‌ی بهبود جریان ترافیک می‌سازد. زیرا که ماشین‌ها و به طور مستقیم رانندگان نیاز به تشویق جهت بهبود جریان ترافیک بوجود آمده از عملکرد خود دارند و متعاقباً مجازاتی هم باید در صورت عملکرد بد به آنها اعمال داشت. همچنبن به دلیل غیرقطعی بودن محیط نیاز است که الگوریتم هربار وضعیت محیط را بررسی کند و یاد بگیرد که عملکرد خود را نسبت به وضعیت‌ها بهبود بخشد.

بهبود جریان ترافیک به دو صورت مایکروسکوپی و ماکروسکوپی مدل می‌شود که یکی وضعیت تک تک خودروها و دومی وضعیت مجموع خودروها را بررسی می‌کند.
کارهای صورت گرفته برای کنترل ترافیک اغلب بر روی بهبود عملکرد چراغ‌های راهنمایی و رانندگی و یا اتخاذ سرعت مناسب در بخش‌های مختلف مسیر تمرکز دارند. برخی رویکردها نیز تلفیقی از این دو را ارائه می‌دهند.

بهبود جریان ترافیک در حوزه‌های شهری و بیرون شهر نیاز به اتخاذ تصمیم‌های متفاوت و متناسب با محیط دارد و در برخی موارد ترکیب این دو رویکرد (در موارد تلاقی بزرگراه‌های بیرون با شهرها) نتایج بهتری داشته است.

\پرش‌بلند
\بدون‌تورفتگی \مهم{کلیدواژه‌ها}: 
جریان ترافیک، یادگیری تقویتی
\صفحه‌جدید
