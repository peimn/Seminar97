\فصل{یادگیری تقویتی}

یادگیری ماشینی یکی از زیرشاخههای اصلی و کاربردی در حوزه هوش مصنوعی و علوم کامپیوتر میباشد. این علم به جستجو و یافتن الگوریتمهایی به منظور ایجاد قابلیت یادگیری در عاملها میپردازد. بهعبارت دیگر هدف یادگیری ماشینی این است که یک سامانه بدون دریافت صریح دستورات، بتواند دادههای ورودی را به صورت اتوماتیک تحلیل و بر اساس تحلیل های بهدستآمده، در مورد دادههای جدید
تصمیمگیری نماید.

\قسمت{تعریف یادگیری تقویتی}

یادگیری تقویتی یک روش یادگیری بدون ناظر می‌باشد و هدف آن یادگیری یک سیاست و نگاشتی از مشاهدات به اعمال، بر مبنای دریافت بازخورد ازمحیط میباشد. این عمل یادگیری را میتوان به صورت جستجوی مجموعهای از سیاستها بهصورت لحظهای که در تعامل با محیط ارزیابی میشوند بیان کرد.

در یک مسئله یادگیری تقویتی با عاملی مواجه هستیم که از طریق سعی و خطا با محیط تعامل کرده و یاد می‌گیرد تا عملی بهینه را برای رسیدن به هدف مورد نظر سیستم انتخاب کند. در این نوع یادگیری هیچ ناظر خارجی وجود ندارد و عامل به تنهایی و به طور مستقل با محیط تعامل کرده، یاد می‌گیرد، تجربه کسب
می‌کند و پاداشی را متناسب با عمل خود دریافت می‌کند.

ایده اصلی روش‌های یادگیری تقویتی از روی الگوی یادگیری رفتار انسان اقتباس شده است. در این
روش، هیچ راهنما و ناظری به صورت مستقیم بر چگونگی آموزش سیستم تمرکز ندارد. الگوریتم یادگیری با بررسی میزان مطلوبیت نتیجه حاصل از یک عمل و به واسطه اهدای پاداش یا جریمه به تمامی اجزایی که در
حصول این نتیجه نقش داشته‌اند، سعی در اعمال آموزش به این سیستم را دارد.

\قسمت{مدل یادگیری تقویتی استاندارد}

در یک مدل یادگیری تقویتی استاندارد، یک عامل از طریق ادراکات و اعمال خود با محیط در ارتباط
است. مطابق شکل ~\رجوع{شکل:مدل یادگیری تقویتی}، در هر مرحله عامل وضعیت جاری $s$ را به عنوان ورودی دریافت می‌کند و عمل $a$ را برای تولید خروجی انتخاب می‌کند.
به‌ازای هر عمل، عامل ارزش عمل خود را به عنوان یک سیگنال پاداش
$r$ به همراه وضعیت جدید سیستم $i$ را دریافت میکند. پاداش دریافتی می‌تواند مثبت یا منفی باشد. سیستم  کنترل اعمال عامل می‌بایست اعمالی را انتخاب کند که مجموع سیگنال‌های پاداش را افزایش دهد. یک عامل
می‌تواند انجام این سیستم کنترلی را در فرآیند تعامل آزمون و خطای خود با محیط یاد بگیرد.

\شروع{شکل}
[t]\centerfig{rl-robot-brain.tex}{1.2}
\شرح{یک مدل یادگیری تقویتی استاندارد \cite{Kaelbling1996}}
\برچسب{شکل:مدل یادگیری تقویتی}
\پایان{شکل}