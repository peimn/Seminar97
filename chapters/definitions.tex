
\فصل{معرفی مسئله‌ی بهینه‌سازی جریان ترافیک}

در صورتی که ازدحام اتومبیل‌ها در خیابان‌ها یا بزرگراه‌ها، بیش از ظرفیت آن‌ها باشد در این صورت مشکل تراکم ترافیکی رخ می‌دهد. در صورتی که یک خیابان یا بزرگراه دچار تراکم ترافیکی نشود در این صورت جریان در آن مسیر، جریان آزاد نامیده می‌شود.
تاخیر به وجود آمده در وسایل نقلیه به عنوان یک نتیجه از ترافیک می‌تواند با محاسبه ساعت خودرو\LTRfootnote{vehicle hours} اندازه‌گیری شود \cite{Jacob2006}. یک ساعت خودرو نشان دهنده رانندگی به مدت ۱ ساعت در بزرگراه است. همچنین می‌تواند به صورت رانندگی با ۶۰ وسیله نقلیه در یک بزرگراه به مدت ۱ دقیقه تفسیر شود. تعداد کل ساعت خودرو را می‌توان با جمع‌آوری مقادیر تمام خودرو‌ها بدست آورد. متریک دیگری زمان تاخیر خودرو\LTRfootnote{vehicle delay time} است که زمان رانندگی اضافی در مقایسه با زمان رانندگی در صورت وجود جریان آزاد است. به عنوان مثال، اگر زمان رانندگی یک وسیله نقلیه، در صورت وجود ترافیک، ۶۰ دقیقه باشد و همان مسیر به صورت جریان آزاد به ۴۵ دقیقه زمان نیاز داشته باشد، زمان تأخیر آن خودرو ۱۵ دقیقه خواهد بود.

همانطور که در مقدمه آمده است، توسعه‌ی خطوط ارتباطی همیشه امکان‌پذیر نیست  بنابراین راه‌حل‌هایی دیگر مورد نیاز است. یکی از راه‌حل‌ها اعمال محدودیت بر روی سرعت می‌باشد که در \cite{Jacob2006} نشان می‌دهد قادر به کاهش تراکم در بزرگراه‌ها می‌باشد.
روشی دیگر به کنترل بهینه‌ی چراغ‌های راهنمایی و رانندگی می‌پردازد \cite{Wiering2000}.

تعیین محدوده‌ی سرعت در بزرگراه‌ها همراه با مشکلاتی می‌باشد که در شکل ~\رجوع{شکل:نمونه ازدحام در بزرگراه} مشاهده می‌شود، ناحیه‌ی خاکستری یک ناحیه اشباع شده در نزدیکی رمپ را نشان می دهد و فلش مسیر مسیر جریان را نشان می دهد. اگر حجم تقاضای ترافیک رمپ در سطح بالا باشد، محدودیت های سرعت را می توان به بخش های بالادست اختصاص داد تا باعث کاهش تراکم شوند.

\شروع{شکل}
[t]\centerfig{highway-stretch.tex}{.75}
\شرح{مثالی از ازدحام در نزدیکی تقاطع در بزرگراه \cite{Walraven2016}}
\برچسب{شکل:نمونه ازدحام در بزرگراه}
\پایان{شکل}

%----------------------------- مقدمه ----------------------------------

\قسمت{مدل‌سازی جریان ترافیک}




