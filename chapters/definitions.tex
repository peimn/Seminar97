
\فصل{معرفی مسئله‌ی بهینه‌سازی جریان ترافیک}

در صورتی که ازدحام اتومبیل‌ها در خیابان‌ها یا بزرگراه‌ها، بیش از ظرفیت آن‌ها باشد در این صورت مشکل تراکم ترافیکی رخ می‌دهد. در صورتی که یک خیابان یا بزرگراه دچار تراکم ترافیکی نشود در این صورت جریان در آن مسیر، جریان آزاد نامیده می‌شود.
تاخیر به وجود آمده در وسایل نقلیه به عنوان یک نتیجه از ترافیک می‌تواند با محاسبه ساعت خودرو\LTRfootnote{vehicle hours} اندازه‌گیری شود \cite{Jacob2006}. یک ساعت خودرو نشان دهنده رانندگی به مدت ۱ ساعت در بزرگراه است. همچنین می‌تواند به صورت رانندگی با ۶۰ وسیله نقلیه در یک بزرگراه به مدت ۱ دقیقه تفسیر شود. تعداد کل ساعت خودرو را می‌توان با جمع‌آوری مقادیر تمام خودرو‌ها بدست آورد. متریک دیگری زمان تاخیر خودرو\LTRfootnote{vehicle delay time} است که زمان رانندگی اضافی در مقایسه با زمان رانندگی در صورت وجود جریان آزاد است. به عنوان مثال، اگر زمان رانندگی یک وسیله نقلیه، در صورت وجود ترافیک، ۶۰ دقیقه باشد و همان مسیر به صورت جریان آزاد به ۴۵ دقیقه زمان نیاز داشته باشد، زمان تأخیر آن خودرو ۱۵ دقیقه خواهد بود.

همانطور که در مقدمه آمده است، توسعه‌ی خطوط ارتباطی همیشه امکان‌پذیر نیست  بنابراین راه‌حل‌هایی دیگر مورد نیاز است. یکی از راه‌حل‌ها اعمال محدودیت بر روی سرعت می‌باشد که در \cite{Jacob2006} نشان می‌دهد قادر به کاهش تراکم در بزرگراه‌ها می‌باشد.
راه‌حلی دیگر کنترل بهینه‌ی چراغ‌های راهنمایی و رانندگی می‌باشد \cite{Wiering2000, Touhbi2017}.

تعیین محدوده‌ی سرعت در بزرگراه‌ها همراه با مشکلاتی می‌باشد که نمونه‌ای از آن در شکل ~\رجوع{شکل:نمونه ازدحام در بزرگراه} مشاهده می‌گردد، ناحیه‌ی سرخ رنگ، ناحیه‌ی دارای ازدحام در نزدیکی رمپ را نشان می‌دهد و پیکان‌های سیاه مسیر ترافیک را نشان می‌دهند. در صورتی که ازدحام ترافیک در تقاطع افزایش یابد، در بخش‌های بالاتر، سرعت را می‌توان محدود نمود تا منجر به کاهش ازدحام گردد.
اما اعمال محدودیت بر روی سرعت، دشواری‌هایی دارد، انتخاب زمان اعمال محدودیت و تعیین بازه‌ای مناسب از خیابان یا جاده جهت اعمال محدودیت، کاری دشوار می‌باشد. 
همچنین تغییرات در سرعت مجاز نباید پرشی و لحظه‌ای باشد، بلکه باید به صورتی باشد که موجب به خطر افتادن امنیت در جاده‌ها نگردد.

\شروع{شکل}
[t]\centerfig{highway-stretch.tex}{.75}
\شرح{مثالی از ازدحام در نزدیکی تقاطع در بزرگراه \cite{Walraven2016}}
\برچسب{شکل:نمونه ازدحام در بزرگراه}
\پایان{شکل}

%----------------------------- مقدمه ----------------------------------

\قسمت{مدل‌سازی جریان ترافیک}
بخش قابل توجی از تحقیقات در حوزه‌ی جریان ترافیک در رابطه با مدل سازی و بهینه سازی جریان ترافیک در بزرگراه ها انجام شده است.
مدل‌سازی‌های ترافیک می‌توانند به دو صورت میکروسکوپی\LTRfootnote{Microscopic} و ماکروسکوپی\LTRfootnote{Macroscopic} انجام گیرند.

در حالی که مدل‌های میکروسکوپی جریان ترافیکی را از نقطه‌ی دید رانندگان و وسایل نقلیه نشان می‌دهند، مدل‌های ماکروسکوپی وضعیت کلی ترافیک  مانند تراکم محلی، سرعت و جریان را در فضای زمان توصیف می‌کنند \cite{Treiber2013}.
در مدل‌های میکروسکوپی رفتار ترافیک به صورت مستقل برای هر خودرو تعریف می‌شود که در این تعریف، سرعت خودرو، موقعیت آن و مشخصات خودرو (مانند؛ حداکثر سرعت خودرو و شتاب خودرو) لحاظ می‌گردد. چنین مدلی، شبیه‌سازی دقیقی از جریان ترافیک به دست می‌دهد ولیکن هزینه‌های پردازش بالایی دارند که غیر قابل اجتناب هستند \cite{Walraven2016}.

از سوی دیگر، مدل های ماکروسکوپی، با استفاده از متوسط سرعت و تراکم در بخش‌های مختلف بزرگراه، به مدل‌سازی می‌پردازند.
از آنجایی که در این مدل تعداد محاسبات ثابت می‌باشد و به تعاداد دقیق خودروها بستگی ندارد، این نوع مدل‌سازی به نسبت مدل میکروسکوپی، دارای هزینه‌ی محاسباتی کمتری می‌باشد \cite{Walraven2016}.
%
%\زیرقسمت{مدل ماکروسکوپی METANET}
%مدل METANET سرعت، تراکم و مقادیر جریان مقطع بزرگراه ها را در شکل بسته محاسبه می کند که بستگی به شرایط ترافیکی فعلی و حجم تقاضای ترافیک رمپ ها و خارج از رمپ ها دارد.

\قسمت{فرمول‌بندی به عنوان فرایند تصمیم‌گیری مارکوف}
می‌توان با استفاده از چارچوب فرآیند تصمیم‌گیری مارکوف، بهینه‌سازی جریان ترافیک را به صورت یک مسئله‌ی تصمیم‌گیری متوالی فرمول‌بندی نمود. و به کمک یادگیری تقویتی به صورت خودکار محدوده‌ی سرعت برای بزرگراه‌ها آموزش داده می‌شود.

\زیرقسمت{تعریف فضا}
فضای بزرگراه را بدین صورت در نظر میگیریم که بزرگراه به بخش‌هایی تقسیم‌بندی می‌کنیم که در بخش در نقطه‌ی زمانی مشخصی ویژگی ترافیکی بزرگراه را تعریف کنیم. کنترل زمانی $Tc =C$ قرار می‌گیرد بدین منظور که کنترل سرعت و اعمال محدودیت تنها زمانی صورت می‌گیرد که $C$ مضربی از $T$، گام‌های زمانی شبیه‌سازی باشد.

\شروع{شکل}
[t]\centerfig{highway-sections.tex}{.60}
\شرح{بزرگراه شامل $N$ بخش \cite{Walraven2016}}
\برچسب{شکل:بخش‌های بزرگراه}
\پایان{شکل}

ما یک بزرگراه یک‌طرفه متشکل از N بخش، همانند شکل ~\رجوع{شکل:بخش‌های بزرگراه} را در نظر می‌گیریم.

$s_{t}$ فضا را در زمان $ct$ تعریف می‌کند:

$$
s _ { 0 } = \left( \frac { v _ { \max } } { v _ { f } } , \frac { v _ { \max } } { v _ { f } } , \frac { v _ { 1 } ( 0 ) } { v _ { f } } , \dots , \frac { v _ { N } ( 0 ) } { v _ { f } } , \frac { k _ { 1 } ( 0 ) } { k _ { j a m } } , \dots , \frac { k _ { N } ( 0 ) } { k _ { j a m } } \right)
$$
$$
s _ { t } = \left( \frac { a _ { t - 1 } } { v _ { f } } , s _ { t - 1 } ( 0 ) , \frac { v _ { 1 } ( c t ) } { v _ { f } } , \dots , \frac { v _ { N } ( c t ) } { v _ { f } } , \frac { k _ { 1 } ( c t ) } { k _ { j a m } } , \dots , \frac { k _ { N } ( c t ) } { k _ { j a m } } \right)
$$

متغیرهای حالت اول و دوم نشان دهنده‌ی محدودیت سرعت فعلی و قبلی برای بزرگراه می‌باشند.

$K_{i}(n)$ نشان دهنده چگالی در بخش $i$ در زمان $nT$ می‌باشد، در حالی که $n$ شاخص گام زمانی است.
متغیر $v_{i}(n)$ میانگین سرعت وسایل نقلیه در بخش $i$ در زمان $nT$ می‌باشد.
متغیر $q_{i}(n)$ نشان دهنده‌ی حجم ترافیک خروجی از بخش $i$ و ورودی به بخش $i+1$ در زمان $nT$ می‌باشد.
متغیرهای $M_{i}$ و $L{i}$ نشان دهنده‌ی تعداد خطوط و طول بخش $i$ می‌باشند.
متغیر $w_{i}(n)$ نشان دهنده‌ی طول صف از تقاطع مربوط به بخش $i$ در زمان $nT$ می‌باشد.
یک پارامتر مهم از مدل تراکم ترافیک مسدود شده‌ی، $k_{jam}$ است، زمانی که تراکم به حداکثر خود نزدیک شده و سرعت نزدیک به صفر گشته است.
سرعت جریان آزاد، $v_{f}$ سرعت مورد نیاز خودروها در زمانی است که ترافیک نبوده و هیچ محدودیتی از طرف سایر وسایل نقلیه تحمیل نشده باشد.

\زیرقسمت{فضای حالت}

\زیرقسمت{تابع پاداش}

\قسمت{یادگیری  سیاست‌ها با استفاده از یادگیری-Q و شبکه‌های عصبی}

\زیرقسمت{الگوریتم یادگیری سرعت محدود}

\زیرقسمت{تقریب تابع مقدار-Q}

