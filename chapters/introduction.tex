

\فصل{مقدمه}

در سال‌های اخیر به منظور روان نگه داشتن جریان حمل و نقل در زمان‌های اوج ترافیک و زمان‌های کمبود منابع، مانند سوخت یا برق، راه‌حل‌هایی هوشمند و علمی پیشنهاد داده شده‌اند.
همچنین مشارکت مهندسین، فیزیک‌دانان، ریاضیدانان و روانشناسان، منجر به درک بهتر رفتار راننده‌گان و در نتیجه بهبود جریان ترافیک\LTRfootnote{Traffic flow} وسایل نقلیه شده است.
این زمینه‌ی کاری جدید بین‌رشته‌ای، در آینده پیشرفت‌های بیشتری را قطعا ایجاد خواهد نمود.
تمرکزها بر روی اپلیکیشن‌های جدید می‌باشند از کمک‌راننده‌های هوشمند گرفته تا رویکردهای هوشمند در جهت بهبود جریان ترافیک و پیش‌بینی دقیق ترافیک.
ترافیک مشکلی است که روزانه اغلب مردم با آن مواجه می‌شوند. ترافیک نه تنها باعث اتلاف زمان می‌گردد بلکه پیامد هدررفت سوخت و ایجاد آلودگی برای محیط زیست را نیز با خود دارد.
در ایالات متحده، درسال ۱۹۸۲، ترافیک باعث هدررفت ۱/۱ میلیارد ساعت زمان بوده و در سال ۲۰۱۱ این هدررفت به ۵/۵ میلیارد ساعت افزایش یافته است \cite{Walraven2016}.
در همان بازه زمانی مصرف سوخت از ۱/۹ میلیارد لیتر به ۱۱ میلیارد لیتر افزایش یافته و در کل هزینه‌ای معادل ۱۲۱ میلیارد دلار در سال ۲۰۱۱ در بر داشته است.

توسعه زیرساخت‌ها و خطوط ارتباطی یکی از اولین راه‌حل‌های ساده‌ی قابل اتخاذ جهت رفع مشکلات مذکور می‌باشد. اما محدودیت‌های موجود، از جمله کمبود زمین مناسب (به خصوص در محیط‌های شهری) و نیز کمبود بودجه، از عوامل بازدارنده‌ی توسعه‌ی زیرساخت و خطوط ارتباطی می‌باشند.

بنابرین بهبود زیرساخت‌های موجود یا حتی بهینه‌سازی نحوه‌ی استفاده از این زیرساخت‌ها و کاهش زمان‌های جابه‌جایی به واسطه‌ی سیستم‌های حمل و نقل هوشمند 
(\LRE{ITS\LTRfootnote{Intelligent Transportation Systems}}) از گزینه‌های موجود به منظور رفع مشکل ترافیک می‌باشند.
و از آنجائی که راه‌حلهای بلندمدت برای اين مشكل مستلزم
سرمايه‌گذاری و فرهنگ‌سازی گسترده است، استفاده از راهکارهای سريع  برای کاهش اين مشكل ضروری به نظر می‌رسند. در اين ميان کنترل بهينه چراغ‌های راهنمايی نقش مهمی را در مديريت و کاهش ترافيک ايفا
می‌نمايد \cite{Smith2013}.
%%
امروزه با
پيشرفت فناوری در علوم کامپيوتر و مهندسی برق و کنترل، با نصب چند  سنسور در تقاطع‌ها و به کارگيری کنترلرهای هوشمند، به راحتی می‌توان مديريت چراغ‌های راهنمايی را به يک سيستم هوشمند سپرد و از به هدر رفتن زمان افراد و سوخت خودروها در پشت چراغ قرمزها و همچنين افزايش آلودگی هوا جلوگيری کرد. چراغ راهنمايی هوشمند، سيستمی است که با توجه به حجم خودروهای ورودی به يک تقاطع همسطح، زمان فازهای مختلف چراغ راهنمايی را به صورت عادلانه مديريت می‌کند. در سال‌های اخير روش‌های يادگيری ماشين ازجمله منطق فازی \cite{Srinivasan2003,Balaji2011}، شبكه عصبی \cite{Srinivasan2006,Chao2008} و يادگيری تقويتی \cite{El-Tantawy2014,Jacob2006} پتانسيل‌های بالايی را برای طراحی کنترلرهای هوشمند چراغ‌های
راهنمايی از خود نشان داده‌اند.

نصب تابلوهای سرعت با قابلیت تغییر عدد نمایش با توجه به وضعیت ترافیک، راه‌حلی دیگر می‌باشد که تاثیری مثبت بر جریان ترافیک داشته است \cite{Papageorgiou2008}. 

چندین الگوریتم کنترل به واسطه‌ی تابلو‌های راهنمایی و رانندگی پویا توسعه یافته‌اند که اغلب این روش‌ها واکنش‌پذیر هستند، به این معنی که محدودیت سرعت، زمانی اعمال می‌گردد که ترافیک تشخیص داده شود.

تابلو‌های پویای راهنمایی و رانندگی از دهه‌ی ۷۰ میلادی کاربرد گسترده‌ای پیدا کرده‌اند و نشان داده است که در مقایسه با تابلوهای ثابت، عملکرد بهتری به منظور مهار ترافیک داشته‌اند \cite{Touhbi2017}.

روش‌های بسیاری به مانند (\LRE{SCOOT, SCATS, PRODYN, OPAC, UTOPIA و RHODES}) با در نظر گرفتن تابلوهای پویا توسعه داده شده و مورد بررسی قرار گرفته‌اند اما این روش‌ها نیاز به یک مدل از پیش تعیین‌شده از محیط دارند.
این در حالی است که با توجه به ذات احتمالاتی و غیرقطعی ترافیک، رویکردی مناسب کنترل آن خواهد بود که با تغییرات ترافیک خود را سازگار سازد و الزامی به تعیین مدلی مشخص از محیط، در آن رویکرد نباشد \cite{Touhbi2017}.

علاوه بر بهبود جریان ترافیک، امنیت و سلامت از دیگر دلایل اعمال محدودیت بر سرعت در جاده‌ها و خیابان‌ها می‌باشند.
تحقیقات نشان داده‌اند که افزایش سرعت منجر به افزایش تصادفات و در نتیجه افزایش مرگ و میر ناشی از آن می‌شود.
و نیز کنترل سرعت با تابلو‌های پویا و متغییر منجر به کاهش تصادفات رانندگی به میزان ۵ الی ۱۷ درصد شده است \cite{Zhu2014}.

در این سمینار هدف نشان دادن تکنیک‌هایی از هوش مصنوعی می‌باشد که نقشی مهم در کنترل محدوده‌ی سرعت به صورت پیشگیرانه بازی می‌کنند.

توسعه‌ی سیستم‌های هوشمند به مانند اتومبیل‌های خودران\LTRfootnote{Autonomouse Cars}، موجب انگیزه‌ی بیشتر به منظور به کارگیری هوش مصنوعی در حوزه‌ی ترافیک و حمل و نقل می‌شود.
اتومبیل‌های خودران باید بتوانند با سرعتی مناسب در حرکت باشند تا فاصله‌ی مجاز با اتومبیل‌های دیگر را حفظ کنند و نیز در محاسبات خود شرایط متغییر ترافیکی را لحاظ کنند و بتوانند علت یا علل افزایش  اتومبیل‌های دیگر در مجاورت خود را پیدا کنند \cite{Walraven2016}.

بدین منظور، اتومبیل باید بتواند به صورت خودکار چگونگی رفع مشکلات را پیدا کند، عکس‌العمل‌های مناسب در مقابل محیط درحال تغییر داشته باشد و سلیقه‌ها و اختیارات کاربران را در تصمیم‌گیری‌های خود نیز لحاظ کند.
با در نظر گرفتن چنین جنبه‌های هوشمندی، به تعریف سیستم نرم‌افزاری هوشمند در این حوزه خواهیم رسید که عامل هوشمند\LTRfootnote{Intelligent Agent} نیز خوانده می‌شود. 

اختصاص محدوده‌ی سرعت به‌خصوص برای انواع مسیر نیازمند سیستم کنترلی تطبیقی می‌باشد و این سیستم باید به صورت پویا بوده و پیش‌بینی‌های ترافیکی را در تصمیم‌گیری‌های خود لحاظ کند.

یادگیری تقویتی\LTRfootnote{Reinforcemenet Learning} امکان تطبیق و خودآموزی از پیش‌بینی‌ها و تجربیات قبلی را به ما می‌دهد. الگوریتم یاد شده به عامل‌های هوشمند (مانند خودروها یا علائم راهنمایی و رانندگی) کمک می‌کند محیط را ببینند از آن یاد بگیرند و براساس یادگیری خود بهترین عکس‌العمل را انتخاب کنند (مانند اعمال محدوده سرعتی مناسب توسط تابلو سرعت مجاز). بنابرین یادگیری تقویتی پتانسیل بیشتری در خصوص بهبود خدمات در طول زمان و در تعامل با محیط دارد.
به بيان ساده‌تر، کنترلر مبتنی بر يادگيری تقويتی از طريق
تعامل هوشمند با محيط ترافيكی که دارای روندهای غيرخطی و اتفاقی\LTRfootnote{Stochastic} است تجربه‌اندوزی کرده و برای رسيدن به اهدافش اعمال لازم را
انتخاب مینمايد.

در این گزارش در مورد استفاده از یادگیری تقویتی، به ویژه \LRE{Q-learning} در کنترل ترافیک تحقیق شده است.
این الگوریتم به صورت خودکار و برپایه‌ی ویژگی‌های مسیر و پیش‌بینی‌ها در خصوص شرایط آتی ترافیکی می‌آموزد که چه زمانی باید محدوده‌ی سرعت در مسیرها اعمال شود تا تراکم کاهش یابد.

روش یادگیری تقویتی علاوه بر سادگی و نداشتن پیچیدگی محاسباتی، در عمل بی نیاز به مدل ریاضی محیط می‌باشد و خاصیت تطبیق پذیری با شرایط محیط و مقاوم بودن در برابر تغییرات محیطی را دارد.	