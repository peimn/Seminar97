
\فصل{کارهای پیشین}

در این بخش، مروری کلی بر مجموعه‌ای از کارهای مرتبط با موضوع بررسی خواهند شد. روش‌هایی که رویکردهای موثر هوش مصنوعی را برای حل مشکلات در حوزه‌های ترافیک و حمل و نقل اتخاذ داشته‌اند.

\قسمت{کنترل جریان ترافیک به کمک روش‌های هوش مصنوعی}
در \cite{Fares2014} به منظور کنترل دستگاه‌های کنترل ترافیک د تقاطع (مانند چراغ یا تابلو) از یادگیری تقویتی بهره برده شده است. 
بهینه سازی جریان ترافیک به عنوان یک پروسه تصمیم‌گیری فرموله شده است، این روش تراکم وسایل نقلیه را به اندازه‌ی تراکم بحرانی حفظ می‌کند، به طوری که جریان بهبود یابد.

یکی دیگر از روش‌های کنترل دستگاه‌های کنترل تقاطع با یادگیری-Q را می‌توان در \cite{Rezaee2012} یافت که تعداد خودروهای عبوری از مسیر فرعی به اصلی را کنترل می‌کند.
روشی مشابه در \cite{Davarynejad2011} بیان شده است که صف حاصل از ترافیک را نیز در محاسبات در نظر می‌گیرد.

از شبکه‌های عصبی نیز در \cite{Wei2002} بهره گرفته شده است که در این گزارش به آن پرداخته نشده است.

بهینه‌سازی چراغ‌های راهنمایی در منطقه شهری نیز اهمیت ویژه‌ای دارد. در \cite{Kuyer2008} به هماهنگی چراغ‌های راهنمایی در منطقه شهری با استفاده از یادگیری تقویتی پرداخته شده است، چراغ‌های راهنمایی را عامل‌های هوشمندی در نظر می‌گیرد که عملکرد خود را با یکدیگر هماهنگ می‌کنند. 

یادگیری تقویتی چندعاملی به منظور کنترل چراغ‌های راهنمایی در \cite{Khamis2014,Khamis2012} مورد بحث قرار گرفته است، و کاهش زمان سفر، افزایش ایمنی و بهبود مصرف سوخت را مورد بررسی قرار می‌دهد. مدل‌سازی کنترل چراغ‌های راهنمایی به عنوان یک سیستم چندعاملی، ممکن است با افزایش تعداد تقاطعات و متقابلا افزایش تعداد چراغ‌های راهنمایی به سیستمی پیچیده تبدیل گردد. در \cite{Abdoos2013} روش کنترل سلسله مراتبی را به منظور برطرف‌سازی این مشکل مورد بحث قرار می‌دهد.

در \cite{Zhao2012} راه‌حل‌های هوشمند مانند شبکه های عصبی را به منظور کنترل چراغ‌های راهنمایی،  مورد بررسی قرار داده است و اذعان دارد که برای کنترل ترافیک با استفاده از سیستم‌های هوشمند، تحقیقات بیشتری لازم است.

در \cite{Yau2017} به بررسی انواع الگوریتم‌های یادگیری تقویتی و عملکرد آن‌ها در کنترل و بهبود استفاده از علائم راهنمایی و رانندگی پرداخته است.

\قسمت{دیگر مشکلات مرتبط با حمل و نقل و قابل برطرف‌سازی توسط یادگیری تقویتی}
علاوه بر کنترل وسایل نقلیه به کمک چراغ‌ها و علائم راهنمایی و رانندگی، از یادگیری تقویتی می‌توان در مسیریابی و هدایت وسایل نقلیه نیز بهره برد. که در \cite{Zolfpour-Arokhlo2014} به این مورد پرداخته شده است.



